\documentclass[a4paper]{article} 


\addtolength{\hoffset}{-2.25cm}
\addtolength{\textwidth}{4.5cm}
\addtolength{\voffset}{-3.25cm}
\addtolength{\textheight}{5cm}
\setlength{\parskip}{0pt}
\setlength{\parindent}{0in}

%----------------------------------------------------------------------------------------
%	PACKAGES AND OTHER DOCUMENT CONFIGURATIONS
%----------------------------------------------------------------------------------------

\usepackage{blindtext} % Package to generate dummy text
\usepackage{charter} % Use the Charter font
\usepackage[utf8]{inputenc} % Use UTF-8 encoding
\usepackage{microtype} % Slightly tweak font spacing for aesthetics
\usepackage[english, ngerman]{babel} % Language hyphenation and typographical rules
\usepackage{amsthm, amsmath, amssymb} % Mathematical typesetting
\usepackage{float} % Improved interface for floating objects
\usepackage[final, colorlinks = true, 
        linkcolor = black, 
        citecolor = black]{hyperref} % For hyperlinks in the PDF
\usepackage{graphicx, multicol} % Enhanced support for graphics
\usepackage{xcolor} % Driver-independent color extensions
\usepackage{marvosym, wasysym} % More symbols
\usepackage{rotating} % Rotation tools
\usepackage{censor} % Facilities for controlling restricted text
\usepackage{pseudocode} % Environment for specifying algorithms in a natural way
\usepackage{booktabs} % Enhances quality of tables
\usepackage{tikz-qtree} % Easy tree drawing tool
\tikzset{every tree node/.style={align=center,anchor=north},
        level distance=2cm} % Configuration for q-trees
\usepackage[backend=biber,style=numeric,
        sorting=nyt]{biblatex} % Complete reimplementation of bibliographic facilities
\addbibresource{ecl.bib}
\usepackage{csquotes} % Context sensitive quotation facilities
\usepackage[yyyymmdd]{datetime} % Uses YEAR-MONTH-DAY format for dates
\renewcommand{\dateseparator}{-} % Sets dateseparator to '-'
\usepackage{fancyhdr} % Headers and footers
\pagestyle{fancy} % All pages have headers and footers
\fancyhead{}\renewcommand{\headrulewidth}{0pt} % Blank out the default header
\fancyfoot[L]{} % Custom footer text
\fancyfoot[C]{} % Custom footer text
\fancyfoot[R]{\thepage} % Custom footer text
\newcommand{\note}[1]{\marginpar{\scriptsize \textcolor{red}{#1}}} % Enables comments in red on margin

%----------------------------------------------------------------------------------------









\begin{document}

%-------------------------------
%	TITLE SECTION
%-------------------------------

\fancyhead[C]{}
\hrule \medskip % Upper rule
\begin{minipage}{0.295\textwidth} 
\raggedright
\footnotesize
Martijn Vermeulen \hfill\\   
4390784 \hfill\\
m.r.vermeulen@student.tudelft.nl
\end{minipage}
\begin{minipage}{0.4\textwidth} 
\centering 
\large 
Error detection in relational data\\ 
\normalsize 
Thesis Proposal\\ 
\end{minipage}
\begin{minipage}{0.295\textwidth} 
\raggedleft
\today\hfill\\
\end{minipage}
\medskip\hrule 
\bigskip

%-------------------------------
%	CONTENTS
%-------------------------------

\section{Introduction}
Here you will clearly introduce the reader to the research area of the thesis. If you want to follow Stanford InfoLab's patented five-point structure for Introductions (thank you Jennifer Widom!), you will simply have to answer the following questions (roughly 5-6 sentences each):
\begin{itemize}
    \item What is the problem? 
    \item Why is it interesting and important?
    \item Why is it hard? (E.g., why do naive approaches fail?)
    \item Why hasn't it been solved before? (Or, what's wrong with previous proposed solutions? How does mine differ?)
    \item What are the key components of my approach and results? Also include any specific limitations.
\end{itemize}

Please provide some references to seminal papers of the area of research you are working in. Google Scholar is your friend. Select papers that are important; an important paper has been published in a good conference, it has a good number of citations and it has had impact in the research world. Avoid as much as possible to cite blogs. They will be gone in some months - your thesis will be out there forever. 

\subsection{Introduction}
% What is the problem? 


% Why is it interesting and important?


% Why is it hard? (E.g., why do naive approaches fail?)


% Why hasn't it been solved before? (Or, what's wrong with previous proposed solutions? How does mine differ?)


% What are the key components of my approach and results? Also include any specific limitations.



%------------------------------------------------


\section{State of the Art}
Here you will build a relatively complete list of papers that already solve the problem you are trying to solve (or a very similar one). If you think that your problem is very novel (i.e., nobody has attacked it before) please think twice. Most probably you are missing something. However, if indeed your problem has never been tackled before, you have to convince the reader how come this is the case. A good state-of-the-art section lists papers, giving a short summary of the contributions of those papers are (i.e., what they offer to the literature). After you list those contributions, you have to say how they fall short (i.e., why they do not solve your problem) and what you think your thesis will do to tackle your problem.

\subsection{State of the Art}


\section{Thesis Objective}
The objective of the thesis is, of course, to solve the problem that you mentioned above. However, some problems are too big to be addresses/sollved within a thesis. In this section try to become a bit more specific on what the objective of this thesis is, rather than simply repeating that you want to solve the problem mentioned in the introduction. We need some more details here.

\subsection*{Optional: Concrete Research Question}
Try to clearly formulate your problem in a strict fashion. Sometimes math can be used here if you have a clear idea of what problem you want to solve. 


\subsection*{Optional: Contributions}
State here in a bullet list, what your contributions are going to be. 

\subsection{Thesis Objective}

\subsubsection{Concrete Research Question}

\subsubsection{Contributions}



\section{Milestones and Deliverables}
How are you going to approach your thesis project? Make a concrete plan and insert a Gantt chart describing the various (high level) tasks that you plan to tackle your research problem and write your thesis. 

%------------------------------------------------

\printbibliography

\end{document}