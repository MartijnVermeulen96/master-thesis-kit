\section{Thesis Objective}
% The objective of the thesis is, of course, to solve the problem that you mentioned above. However, some problems are too big to be addresses/solved within a thesis. In this section try to become a bit more specific on what the objective of this thesis is, rather than simply repeating that you want to solve the problem mentioned in the introduction. We need some more details here.

The thesis objective starts with an empirical case study \& comparison of state of the art techniques. In order to do so, 

% Empirical case study & comparison
%
%   |
%   v
%
% Create interactive tool to do stuff with (using abstraction layer like in REDS)
%
%   |
%   v
%
REDS \cite{Mahdavi2019-pk}
% Create data profiles (similar to REDS) to estimate performance of tools on unseen data + create profile repo
% 
%   |
%   v
% 
% Cluster profiles to see what data error detection problems are similar, and what areas might have missing parts
% 
%   |
%   v
% 
% Set future directions based on complete overview


\subsection*{Concrete Research Question} % Optional: 
% Try to clearly formulate your problem in a strict fashion. Sometimes math can be used here if you have a clear idea of what problem you want to solve. 

\begin{outline}
    \1 RQ: How to choose a fitting error detection algorithm for a specific relational dataset?
        \2 RQ1: What is the current state of the art and what is the performance of these tools?
        \2 RQ2: Is it possible to create an extensive data profile to estimate performance on unseen datasets?
        \2 RQ3: Will these data profiles provide more explainability of error detection tools?
        \2 RQ4: What are the future research directions based upon these profile findings?
\end{outline}


\subsection*{Contributions} % Optional: 
% State here in a bullet list, what your contributions are going to be. 

\begin{outline}
    \1 Empirical case study \& comparison
        \2 Objective and accurate testing of different tools against benchmark datasets
        \2 Cell error \& tuple error scores
        \2 F1-score, Precision \& Recall
        \2 Incorporate other settings, like training data size, runtime and human interaction
    \1 Interactive tool to automate execution of different error detection tools
    \1 Data profile repository for further analysis
        \2 Analysis on the profiles and explanations on clusters of profiles
    \1 Recommendations on future directions
\end{outline}