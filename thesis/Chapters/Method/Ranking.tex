
% RQ3
\section{Ranking}
\label{sec:toolranking}
% Most important:
% Finding the top performing tool is most important, then low MSE for that top performing
% Confidence in automation
% 

Now that a predictor for future error detection tool performance is introduced, the goal of an estimated ranking of tools with respect to their performance will be discussed.
~\\As the absolute F1-score is helpful to know with a single tool at your disposal, a ranking or recommendation of tools and their strategies would be of greater use for the end-user.
~\\The question proposed here is to see if such a ranking could be made, ranking the better performing tools higher in the ranking. This also allows for other metrics to experiment with. The regressors (estimators) used in section \ref{sec:performanceprediction} could now also be optimized using information retrieval ranking scores. This might result in different distinguishing features found by the regression models and shows the possible flexibility of these models (if it is possible to get good rankings).

\subsubsection{Method of ranking}

\subsubsection{Performance measure of ranking}

\todo{Use metrics from Information Retrieval}
~\\Discounted cumulative gain 