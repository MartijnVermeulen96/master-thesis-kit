\section{Research Questions}
\label{sec:researchquestions}
Resulting from this motivation, the following research questions will tried to be answered in this thesis:

\paragraph{\textit{Main Research Question:}} 
~\\\textbf{How to choose a fitting error detection algorithm for a specific relational dataset?}

~\\With the following sub-questions:
\paragraph{RQ1} What is the current state of the art and what is the performance of these tools?

\paragraph{RQ2} Is it possible to create an extensive data profile to estimate performance on unseen datasets?

\paragraph{RQ3} Is it possible to generate a ranking of tools according to their performance on unseen datasets?

\paragraph{RQ4} Do these data profiles provide more interpretability of error detection tools?

\section{Contributions}
\label{sec:contributions}
By answering the research questions above, the following contributions have been made in this research:

\begin{itemize}
    \item An error detection framework with 6 different error detection tools
    \item A comparison study of error detection algorithms for relational data
    \item A suggestion system for error detection tools and configuration for unseen relational datasets
    % \item Method of interpretability 
\end{itemize}

\section{Outline}
\label{sec:outline}
The outline of this thesis is as follows. 
First, in chapter \ref{chap:background}, the background on error detection in relational data is given. Besides the foundational information on error types, scores and error detection benchmarking, state of the art tools and methods will be covered. Also, the background on dataset profiling and interpretability with respect to error detection will be given. 
In chapter \ref{chap:methodology}, the methodology of this research is highlighted, in order to answer the research questions given in \ref{sec:researchquestions}. Specific used datasets, error detection tools and machine learning techniques will be covered.
Then, in chapter \ref{chap:results}, the results of the research will be presented. 
In chapter \ref{chap:discussion} the discussion on the research is provided and lastly in chapter \ref{chap:conclusion}, conclusions and future work will be discussed.