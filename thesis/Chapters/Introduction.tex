\chapter{Introduction}
\label{chap:introduction}
% Story on data cleaning, to introduce the subject

\section{Errors}
\label{sec:errors}
% What are errors?
Error detection is the first step in the data cleaning process.

\blockquote{Data cleaning is the process of \textbf{\textit{detecting}} and correcting (or removing) corrupt or inaccurate records from a \textbf{record set, table, or database} and refers to \textbf{identifying incomplete, incorrect, inaccurate or irrelevant parts} of the data and then replacing, modifying, or deleting the dirty or coarse data.}

The goal of error detection is to identify incomplete, incorrect, inaccurate or irrelevant parts of the data.

% What are errors in relational data sources + what are relational data sources?

% What is the problem? 

\section{Motivation}
\label{sec:motivation}
% Why is it interesting and important?
%% Industry & academics

%% Focus on the automatic error detection, expert-free, no rules etc...

%% Costly to run all

%% Interpretability use case / explainable AI

\section{Research Questions}
\label{sec:researchquestions}
Resulting from this motivation, the following research questions will tried to be answered in this thesis:

\paragraph{Main Research Question} How to choose a fitting error detection algorithm for a specific relational dataset?

~\\With the following sub-questions:
\paragraph{RQ1} What is the current state of the art and what is the performance of these tools?

\paragraph{RQ2} Is it possible to create an extensive data profile to estimate performance on unseen datasets?

\paragraph{RQ3} Is it possible to generate a ranking of tools according to their performance on unseen datasets?

\paragraph{RQ4} Do these data profiles provide more interpretability of error detection tools?


\section{Outline}
\label{sec:outline}
